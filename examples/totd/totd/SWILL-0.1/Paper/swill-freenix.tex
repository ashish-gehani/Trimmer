%template for producing IEEE-format articles using LaTeX.
%written by Matthew Ward, CS Department, Worcester Polytechnic Institute.
%use at your own risk.  Complaints to /dev/null.
%make two column with no page numbering, default is 10 point
%\documentstyle[times]{article}
\documentstyle[twocolumn]{article}
%\pagestyle{myheadings}

%set dimensions of columns, gap between columns, and space between paragraphs
%\setlength{\textheight}{8.75in}
\setlength{\textheight}{9.0in}
\setlength{\columnsep}{0.25in}
\setlength{\textwidth}{6.45in}
\setlength{\footheight}{0.0in}
\setlength{\topmargin}{0.0in}
\setlength{\headheight}{0.0in}
\setlength{\headsep}{0.0in}
\setlength{\oddsidemargin}{0in}
%\setlength{\oddsidemargin}{-.065in}
%\setlength{\oddsidemargin}{-.17in}
%\setlength{\parindent}{0pc}

%I copied stuff out of art10.sty and modified them to conform to IEEE format

\makeatletter
%as Latex considers descenders in its calculation of interline spacing,
%to get 12 point spacing for normalsize text, must set it to 10 points
\def\@normalsize{\@setsize\normalsize{12pt}\xpt\@xpt
\abovedisplayskip 10pt plus2pt minus5pt\belowdisplayskip \abovedisplayskip
\abovedisplayshortskip \z@ plus3pt\belowdisplayshortskip 6pt plus3pt
minus3pt\let\@listi\@listI} 

%need an 11 pt font size for subsection and abstract headings
\def\subsize{\@setsize\subsize{12pt}\xipt\@xipt}

%make section titles bold and 12 point, 2 blank lines before, 1 after
\def\section{\@startsection {section}{1}{\z@}{24pt plus 2pt minus 2pt}
{12pt plus 2pt minus 2pt}{\large\bf}}

%make subsection titles bold and 11 point, 1 blank line before, 1 after
\def\subsection{\@startsection {subsection}{2}{\z@}{12pt plus 2pt minus 2pt}
{12pt plus 2pt minus 2pt}{\subsize\bf}}
\makeatother

\newcommand{\ignore}[1]{}
%\renewcommand{\thesubsection}{\arabic{subsection}.}

\begin{document}



%don't want date printed
\date{}

%make title bold and 14 pt font (Latex default is non-bold, 16 pt)
\title{
{\large \em *** DRAFT : To appear in Freenix'02 ***} \\
{\Large \bf   SWILL: A Simple Embedded Web Server Library}}

%for single author (just remove % characters)
\author{Sotiria Lampoudi and David M. Beazley \\
{\em Department of Computer Science} \\
{\em University of Chicago }\\
{\em Chicago, Illinois 60637 }\\
{\tt \{slampoud,beazley\}@cs.uchicago.edu }}

\maketitle

\thispagestyle{empty}
\begin{abstract}

{\em
We present SWILL, a lightweight programming library that adds a simple
embedded web server capability to C and C++ programs.  Using SWILL, it
is possible to add Internet accessibility to programs that are poorly
matched to more traditional methods of web programming such as CGI
scripting or web server plugin modules.  SWILL also makes it easy for
programmers to add web-based monitoring, diagnostics, and debugging
capabilities to software not normally associated with internet
programming.  
We like to think of SWILL as an attempt to turn
the problem on its head: traditionally, the web server came first, the 
``programs'' later; in our approach, the application is written first, and the 
server integrated last. For some types of applications, this 
approach is far more painless.
In this paper, we provide an overview of the SWILL 
library and describe how we have used it to provide web access to a 
variety of applications including scientific simulation software, 
a compiler, and a hardware emulator for teaching operating systems.
}

\end{abstract}

\section{Introduction}

With the growth and popularity of the web, many software developers
are interested in building web-based interfaces to their applications.
However, much of the effort involved in doing this seems to be spent figuring
out how to integrate an application into an existing web server using
CGI scripts, server plugin modules, or some sort of middleware
layer. Although it is certainly possible to implement a web interface
in this manner, a programmer may have to contort their application or
implement a complicated runtime environment to make it work. To
further complicate matters, there are certain applications in which
web access would be useful, but which do not really fit into the standard mold
of an internet application.  Examples might include simulators, long
running scientific programs, and compilers. 

Instead of figuring out how to integrate an application with a
web server, an alternative approach might be to flip the
whole situation around and to ask what it would take to
add a web server to an application as a programming library.  If a web
server could be added as a library, it might greatly simplify the task
of creating web-accessible programs.  
For instance, since the the web
server would be part of the application itself, there would be no need
to worry about the installation and configuration of a complicated web
server environment.  A library would also make it easier to add web
capabilities to programs that are not really intended to be internet
applications, but where a web interface would be useful in providing
remote accessibility or as a simple diagnostics tool.

In this paper, we describe SWILL (Simple Web Interface Link Library).
SWILL is a lightweight library we have developed that makes it easy to
add a web server to C/C++ programs. It consists of a handful of C
functions that are inserted into an application to make it web
accessible. The implementation is intentionally minimal and is
primarily intended for developers who would like to create a web
interface without a lot of hassle.

SWILL was initially written for use with high performance scientific
simulation software \cite{beazley1}.   However, the implementation is generic enough to be
used in a variety of other applications--limited only by the
programmer's imagination.   For example, we have used SWILL in a
compiler project to help with the debugging of parse tree data.
It has also been used to monitor the internals of a hardware simulator
for teaching operating systems.

The rest of this paper provides a brief overview of SWILL followed by
some examples of how we have used the library.  At the end, we provide
a brief overview of related work and future plans.

\section{Library Introduction}

The best way to introduce SWILL is with a simple example.  Although
the library contains about 25 functions, only a handful are needed to
get started as shown in this ``Hello World'' example:

\begin{verbatim}
#include <swill.h>
void hello(FILE *f) {
   fprintf(f,"Hello World!\n");
}

int main() {
   swill_init(8080);
   swill_handle("hello.txt",hello,0);
   while (1) {
      swill_serve();
   }
}
\end{verbatim}

In this example, the web server is activated by calling {\tt
swill\_init()} with a TCP port number.  One or more documents are then
registered with the server using {\tt swill\_handle()}.  {\tt
swill\_serve()} is then used to wait for a connection.  When a
connection is received, an appropriate handler function is invoked
depending on the URL. In this case, a request for the document ``{\tt
hello.txt}'' invokes the {\tt hello()} function and produces the
output that you expect.

If an application wants to perform other work in addition to checking
for connections, a non-blocking polling function is used.  
For example:

\begin{verbatim}
while (1) {
   swill_poll();  /* Requests? */
   ...
   /* other work */
   ...
}
\end{verbatim}

This approach allows the web interface to be easily inserted into
applications that provide their own event or computation loops.  
For example, a
scientific simulation might call {\tt swill\_poll()} at selected
intervals during computation, but spend the rest of its time crunching
numbers.

There are no restrictions on the type of output a SWILL handler
function may produce.  However, the document type is implicitly determined
by the suffix supplied to the {\tt swill\_handle()} function. For example,
if a function produces HTML, the following code would be used:

\begin{verbatim}
void hello(FILE *f) {
   fprintf(f,"<HTML><BODY>\n");
   fprintf(f,"<b>Hello World!</b>\n");
   fprintf(f,"</BODY></HTML>\n");
}

int main() {
   ...
   swill_handle("hello.html",hello,0);
   ...
}
\end{verbatim}

Similarly, the following code produces a PNG image using the freely available GD library \cite{gd}:

\begin{verbatim}
void
image(FILE *f) {
  int black,white;
  gdImagePtr im;
  
  im = gdImageCreate(64,64);
  black = gdImageColorAllocate(im,
                0,0,0);
  white = gdImageColorAllocate(im,
                255,255,255);
  gdImageLine(im,0,0,63,63,wht);
  ...
  gdImagePng(im,f);
  gdImageDestroy(im);
}

int main() {
   ...
   swill_handle("image.png",image,0);
   ...
\end{verbatim}

Since a programmer may want to reuse the same handler function for different web pages, SWILL allows an
optional pointer to be passed to the handler functions.  This pointer is used to pass
application specific data or an object to the handler.  For example, an application that created
various types of data plots might look roughly like this:

\begin{verbatim}
void
make_plot(FILE *f, void *clientdata) {
  Plot *p = (Plot *) clientdata;
  ...
  // Generate plot
  ...
  write_plot(p,f);
}

int main() {
  ...
  e = new energy_plot();
  d = new density_plot();
  ...
  swill_handle("energy.png",make_plot,e);
  swill_handle("density.png",make_plot,d);
   ...
\end{verbatim}

Although SWILL is primarily intended for dynamic content generation, 
it can also deliver individual files or files from a user-specified 
directory. For example, if a programmer wanted to register a specific 
file with the server, they would do this:

\begin{verbatim}
swill_file("foo.html","./htdocs/foo.html");
\end{verbatim}

\noindent
Similarly, a directory of files is registered as follows:

\begin{verbatim}
swill_directory("./htdocs")
\end{verbatim}

\noindent
When a directory is registered, SWILL delivers files much like a
traditional web-server.

In certain applications, a programmer might want to receive HTTP query
variables as input parameters (as might be supplied from an HTML
form).  SWILL automatically parses HTTP query strings in both GET and
POST requests.  To access query variables as strings, the following
function is used:

\begin{verbatim}
char *swill_getvar(const char *name);
\end{verbatim}

\noindent
However, a more convenient way to get form variables is to use {\tt
swill\_getargs()} as shown in this example:

\begin{verbatim}
void adder(FILE *f) {
   double x,y;
   if (!swill_getargs("d(x)d(y)",&x,&y)) {
     fprintf(f,"Missing values!\n");
     return;
   }
   fprintf(f,"%g + %g = %g\n", x,y,x+y);
}
\end{verbatim}

\noindent
The argument to {\tt swill\_getargs()} is a format string that
specifies the types and names of form variables to be converted.  If
available, the variables are decoded, placed into C variables, and a
success code returned.

\section{Concurrency and I/O}

SWILL is a single-threaded server that does not rely upon concurrency
mechanisms such as forking or multithreading.  This limitation is by
design and is related to the goal of having a server that could be
easily embedded into a variety of specialized applications such as
parallel scientific codes, hardware emulators, and so forth.  In these
situations, the use of concurrency can introduce serious
reliability and portability problems.  For instance, an application
may not be thread-safe, making it impossible to reliably execute a
handler function in parallel with normal execution.  Similarly,
forking may be impractical in applications with heavy resource
utilization or which rely upon interprocess communication (e.g.,
message passing).

Because of the single-threaded execution model, the implementation of
SWILL relies entirely upon non-blocking I/O with timeouts.  This
prevents the server from indefinitely blocking the application in the
event of bad connections and missing data. For instance,
if a connection is made, but no HTTP headers are received, SWILL
automatically closes the connection after a timeout and returns.  
The timeout is
fully configurable by the user and can be set to only a few seconds if
desired.

The I/O for handler functions relies upon a temporary file created
with {\tt tmpfile()}.  When requests are serviced by handler
functions, output is placed into this file.  When a handler function
has finished execution, the file contents along with HTTP headers are
sent back to the client.  Normally, SWILL simply passes a corresponding
{\tt FILE *} object to handler functions so that they can perform I/O.
As an optional feature, SWILL can also capture standard output.
This is enabled in {\tt swill\_handle()} by prefixing the document name
with {\tt stdout:} as follows:

\begin{verbatim}
swill_handle("stdout:foo.html",foo,0);
\end{verbatim}

In this case, all I/O operations on {\tt stdout} are captured for the duration
of the handler function.  This capture is sufficiently powerful to allow 
other programs to be executed using {\tt system()} and to have the output
of those programs redicted to a web page.  For example, the following handler
function would capture the output of a system command:

\begin{verbatim}
void listfiles() {
    system("ls -l");
}
...
swill_handle("stdout:files.txt",listfiles,0);
\end{verbatim}

\section{Security and Reliability}

SWILL is not appropriate for use in applications that require a high
degree of security since no support for SSL or cryptographically
secure user authentication is provided.  However, the library does
provide a few simple security features to restrict access.  First,
basic HTTP user authentication is available by registering names and
passwords like this:

\begin{verbatim}
swill_user("dave","iluvschlitz")
\end{verbatim}

Second, IP filters can be used to disallow or allow connections from
specific IPs or ranges of IPs.  For example:

\begin{verbatim}
swill_allow("127.0.0.0");  
swill_deny("128.135.11."); 
swill_deny("128.135.11.8");
swill_deny("");    
\end{verbatim}

SWILL also allows users to register a log file in which requests will
be recorded and which can be monitored to check for suspicious
activity.

Due to the lack of concurrency, a SWILL application may be vulnerable
to a denial of service attack.  
However the library does take reasonable precautions to allow an application to make progress. 
As mentioned in the previous section, all I/O operations involve
non-blocking system calls with timeouts.  Therefore, it is not
possible for a client to indefinitely block execution by keeping the
connection open without transmitting any data.   SWILL is also quick to close
connections if it detects malformed data such as bad HTTP headers or garbled
input.   We have considered the possibility of automatically blocking
IP addresses that repeatedly send bad requests.  
However, this is not implemented
at this time.  Given that IP filters can be used to block access, a user
already has the means to restrict access to a set of known hosts.

Finally, since SWILL is embedded in an application, it is certainly
possible for bad programming to break the server.  For instance, a
poorly written handler function could enter an infinite loop or start
a computation that exceeds available machine resources.  In this case,
the application would become unresponsive and would probably die.
SWILL does not take any steps to prevent such problems.  However, these
can be anticipated with a certain amount of common sense, error
checking, and having an understanding of the underlying
execution model of the application.

\section{Parallelized SWILL}

A very useful feature of SWILL is that it supports SPMD-style parallel
applications that utilize MPI (MPICH is currently supported)
\cite{mpich}.  This allows it to be used on Beowulf clusters and large
parallel machines.  If used in this style, every node calls {\tt
swill\_poll()} in parallel which results in a global synchronization.
If an incoming request is received, it is forwarded to all of the
nodes which then execute the handler function in parallel.

Using SWILL in this setting is no more difficult than in the single
processor setting; the HTTP client connects to the master node of the
computation, issues a request and receives sorted output collected
from all nodes.  Under the hood the implementation is also quite
simple. The master node ({\tt MPI\_Rank $==$ 0})
receives requests and broadcasts them to all of the other nodes.
Each node runs the handler function in parallel after which
the content is collected by the master node and served in a coherent
manner to the HTTP client. 

In parallel scientific programming performance is the foremost consideration.
To evaluate the performance of {\tt swill\_poll()} we distinguish three cases: a) the case in
which there is no pending HTTP request, b) the case where there is a
pending request for a file or authentication, and c) the case where
there is a request for dynamic content.
\begin{itemize}
{\item[a)]
This case is quite simple. If there is no pending HTTP request, the
master node communicates a null request to all nodes, and the host
code continues execution immediately after the call to {\tt
swill\_poll()}. There is one synchronization in this case, for the call
to broadcast the null request.
}
{\item[b)]
This case is also handled with an overhead of one
synchronization. The master node identifies the pending request as one
that can be served by it alone, serves it and broadcasts a null
request. 
This also reveals one of our underlying assumptions, namely that the SPMD 
program is running on a shared
filesystem of some sort, so that it would make no sense to return {\em
n} (for {\em n} nodes) copies of the requested file. This behavior is
easy to get around. If it is desirable that a copy of the requested
file be returned from {\em each} node, the user can just write a
function that reads the file in and prints it to stdout or uses {\tt
swill\_printf()}.
}
{\item[c)]
This is a somewhat more expensive operation. 
The master node broadcasts the pending request to all
nodes, who parse it and execute the appropriate function. Then each
node transmits the result of its execution to the master. Next, the
master node serves the HTTP response and resumes execution of the host
code, while the back-end nodes resume computation immediately.
All in all, this case has a cost of one broadcast plus {\em
n} (for {\em n} nodes) communications.
}
\end{itemize}

In our quest for simplicity we have left it up to the
user to produce output that denotes what process each segment of the
output is coming from -- SWILL merely orders it in rank order and
serves it.

One interesting use of the parallel capability of SWILL is in overcoming
limitations -- whether due to architecture or policy -- imposed by the
administrators of clusters and supercomputing centers.  Often it is
not possible to access the backend computational nodes of a
supercomputer or cluster in order to query the state of one's
execution -- at least not through a normal shell.  When an application
embeds a web server, all that is required is knowledge of the master
node and access to an unprivileged port.  Web requests can then be
easily translated into operations that execute on each node of the system.

\section{Applications}

SWILL has primarily been developed as a tool for remote process monitoring,
debugging, and diagnostics.   This section describes some applications in
which we are using the library.

\subsection{Scientific simulation monitoring}

The motivating application for most of SWILL's development has been
that of monitoring long-running scientific simulations.  These
programs are typically non-interactive batch jobs that provide little
in the way of user feedback.  However, to make sure a program is running
correctly, a scientist may want to periodically monitor the state of
their programs.  For example, they might want to check for numerical
instabilities or to see how far an experiment has progressed.

To do this, a simulation can be modified slightly by implementing a few handler
functions and inserting {\tt swill\_poll()} calls into selected places
in the simulation loop.  For example:

\begin{verbatim}
/* Initialize SWILL */
swill_init(3737);
/* Register a few handlers */
swill_handle(...);
swill_handle(...);
...

for (i = 0; i < nsteps; i++) {
    compute_forces();
    integrate();
    boundary_conditions();
    redistribute_data();
    if (!(i % output_freq)) {
        write_output();
    }
    /* Check for connections */
    swill_poll();
}
\end{verbatim}

Although this is only a simple example, this technique is easily extended
to provide a variety of advanced monitoring capabilities.  For
instance, if a graphics library is available, the web interface can be
used to generate on-the-fly plotting and data visualization.  Web access
can also be provided to temporary files and other debugging output as
the simulation runs.  Using HTTP query variables and forms, a
scientist could even alter various simulation parameters, enable
diagnostic features, temporarily suspend computation, and so forth.

\subsection{Compiler parse tree browsing}


\begin{figure*}[t]
\begin{picture}(400,380)(0,0)
\put(30,-90){\special{psfile = swig.ps hscale = 70 vscale = 70}}
\end{picture}
\caption{Parse-tree browsing in SWIG}
\label{swig}
\end{figure*}

As a more unusual example, we have used SWILL to provide a web
interface to SWIG, a compiler for creating scripting language
extensions \cite{swig}.  One of the challenges of compiler
implementation is that of creating, traversing, and managing parse
tree data.  For the purposes of debugging, it is fairly common to dump
the parse tree into a text file where it can be examined.
Unfortunately, even for small input files, this might generate a lot
of output since a parse tree might contain hundreds to thousands of
nodes.  This makes it difficult to find the specific information of
interest.

As an alternative to dumping the parse tree to a file, a more convenient way
to examine the parse tree data is to run SWIG using a special {\tt -browse}
option like this:

\begin{verbatim}
$ swig -browse -c++ -python example.i
SWIG: Tree browser listening on port 4908
\end{verbatim}

\noindent
In this mode, the compiler enters a web-server mode after all parsing and code generation
stages have been completed.  Then, by pointing a browser at the appropriate port number, it is
possible to point-and-click through internal parse tree data. A sample screenshot of this
interface is shown in Figure \ref{swig}.  Unlike the information in a text dump,
the web interface provides a more more detailed picture of how information is organized in the
compiler.  For instances, pointers are represented by hyperlinks and clicking on a link is
essentially the same as dereferencing a pointer in C.   The web interface also provides
access to compiler symbol tables, type tables, and other information.

At first glance, the idea of a web-enabled compiler sounds crazy.
However, this capability greatly simplifies debugging and development
of the compiler. Furthermore, a web browser interface works perfectly
well as a simple data exploration tool and it was extremely easy to
implement---requiring only a half day of effort and a few hundred
lines of code.  In comparison, the development of a customized tree
browser using a GUI toolkit such as Tcl/Tk would have been a much more
involved project, would have greatly complicated the configuration of
the compiler, and would have provided little if any extra
functionality.

\subsection{Operating systems project}

At the University of Chicago, the course in operating systems requires
students to implement a simple Unix-based kernel that runs within an
emulated hardware environment.  For emulation, we use a modified
version of Yalnix, an emulator originally developed by Dave Johnson at
Rice University \cite{yalnix}.
In this project, students are given eight
weeks to implement a kernel from scratch including boot loading,
virtual memory management, I/O device drivers, processes, and
interprocess communication.  Kernels are implemented in C and
typically consist of a few thousand lines of code.

One of the problems of working with the emulator is that debugging is
extremely difficult.  For the most part, the only available
diagnostics are an optional hardware trace file and the output
of print statements included in the student's implementation.
Unfortunately, this information is often incomplete--making it very
difficult to reconstruct the system state that might be the source of
a problem.

As an experiment, we have recently used SWILL to add a web-interface to the
Yalnix emulator. Internally, Yalnix relies heavily on advanced
features of signal handling on Solaris.  Specifically, user-mode
programs execute natively on the SPARC whereas the student kernel runs
in response to signals such as {\tt SIGSEGV} and {\tt SIGALRM} (which
are transformed into ``hardware interrupts'').  To instrument the emulator with
a web server, we implemented a number of handler functions,
initialized the server on startup, and placed a {\tt swill\_poll()}
call into the {\tt SIGALRM} handler that is used for internal timing of the emulator.

Using the web interface, it is possible to directly connect to the
emulated hardware.  Available information includes the current status
of all hardware registers, page table settings, and the contents of
physical memory pages.  It is also possible to pause execution and to
obtain traces of recent hardware operations (the history of memory
mapped I/O ports, registers, and interrupts).  More importantly, the
web interface allows you to observe system behavior that is nearly
impossible to obtain otherwise.  For instance, by watching page tables
you can easily spot kernel memory leaks and other inefficiencies in
the implementation.  

\section{Discussion of Related Work}

Internet programming is obviously a huge topic, making a detailed
comparison of SWILL to related work difficult.  Overall, we feel that
SWILL differs from other work in a number of respects.  First, a
considerable amount of attention has been given to programming
techniques such as CGI scripting, web server modules, server pages, 
and programming environments for building Internet applications
\cite{cgi,apache, csp, hsp, zope}.  Although these techniques are successfully
used to incorporate programming libraries and other applications into
a larger Internet framework, SWILL has a somewhat different focus than
this. Instead of trying to build Internet applications, SWILL is
mostly concerned with providing Internet access. This may be a subtle
point, but the applications for which SWILL is the most useful are not really
designed for the Internet---Internet access is merely an add-on feature that
can enhance them. 

SWILL might also be compared to work in distributed
computing.  For instance, SOAP and XML-RPC are often mentioned as
mechanisms for adding internet access to applications \cite{soap,xmlrpc}.
In fact, toolkits such as gSOAP can be used to simplify the
integration of an application into such an environment \cite{gsoap}.
The problem with these approaches is that they are mostly focused on
the problem of turning an application into some sort of pluggable
network service to be used within a complicated middleware layer.  SWILL,
on the other hand, is much more lightweight and is oriented more towards
end-users.  For instance, users merely connect to the server and
are presented with application-specific information in a format that is
easy to use and manipulate.   There is no hidden application framework
or network layer at work.

Finally, SWILL is closely related to domain-specific efforts in
providing remote access to applications.  For instance, in scientific
computing, a lot of attention has been given to the area of
``computational steering'' \cite{vetter2}.  
One of the primary goals
of steering research is to provide fine-grained interactivity and user
feedback to scientific software that is normally
batched-oriented and non-interactive.  Traditionally, this work has
relied upon customized network protocols, complicated client
software, and high-end hardware such as graphics workstations.
(For a detailed discussion of computational steering we refer the reader to the excellent article \cite{vetter}.)

In some cases, application frameworks
may provide web access for this purpose. The Cactus code \cite{cactus, 
cactuspaper}, a modular scientific programming framework,  is 
such an example (the CactusConnect/HTTP and CactusConnect/HTTPExtra ``thorns'' -- modules -- 
 are supposed to provide HTTP access to a cactus application).  In such cases, 
however, the web
access features are not usable as a stand-alone library. If one wants to have 
web access, one is forced to buy into a framework with all the pain and risks 
that entails. SWILL tries to avoid this by focusing
exclusively on the problem of web access.

\section{Implementation Details}

SWILL is implemented entirely in ANSI C and consists of about 2500
semicolons.  Most of the implementation (1500 semicolons) simply provides a small set of
generic data structures (hashes, lists, strings, etc.) that are used elsewhere.
The library itself requires minimal memory overhead.
However, all of the generated web pages involve internal buffering.
Therefore, memory use is directly proportional to the size of
generated web pages.  Clearly the library would be inappropriate for
serving huge amounts of data.  However, this was not a design goal.

The performance overhead of using SWILL depends on frequency of
polling (and obviously the number of incoming connections).  On a
single CPU, {\tt swill\_poll()} is nothing more than a thin wrapper
around the {\tt select()} system call.  With MPI, polling requires a
barrier synchronization across processors.  This is obviously more
expensive and careful consideration must be given to parallel
applications.

For networking, SWILL relies upon the HTTP/1.0 protocol.  Although this is
less powerful than HTTP/1.1, it is easier to implement and perfectly
well suited for most situations.

\section{Limitations}

SWILL was primarily designed as a tool for building quick-and-easy web
interfaces to C/C++ programs.  Its single threaded execution would be
inappropriate for a high-traffic web site and you probably would not
want to use it as the basis of a large internet application.
Similarly, internal buffering and other aspects of the implementation
make the server inappropriate for delivering very large amounts of
data.  The server is also unable to support data streaming or any sort
of application in which the HTTP connection would be kept alive over a
prolonged period.

Although SWILL provides some basic security mechanisms, it would not
be appropriate for applications in which security was critical.  It
should also be added that firewalls and other security mechanisms may
prevent users from accessing a server if access to user TCP ports is
blocked.  Obviously, SWILL cannot address these problems of social
engineering.

\section{Future Plans}

In our own work, SWILL has proven to be remarkably simple and
effective to use.  Therefore, we have every intention of preserving the minimal
nature of the implementation.   However, it may be interesting to provide
alternative interfaces to C++ and Fortran.  Since SWILL does
not rely upon anything more than a few simple functions and standard I/O operations,
it would be relatively easy to implement handler functions with a slightly
different calling convention.  For example, in C++, SWILL might encapsulate I/O
in an iostream object instead of a {\tt FILE *}.

We have also considered the idea of allowing each SWILL-server to
``phone home'' to a user-defined master server.  If a user was running
many different web-enabled applications, this scheme might make it
easier to keep track of where they are running.  For example, a user
could simply connect to the master server and jump to a specific
application from there.

Obvious improvements could be made to the underlying HTTP protocol such as support
for HTTP/1.1, secure sockets, and digest-based user authentication.  However,
supporting such features would introduce a lot of extra complexity and would probably
offer only marginal benefits in return.

\section{Availability}

SWILL is freely available under a LGPL licence.  More information is available at:

\begin{center}
{\tt http://systems.cs.uchicago.edu/swill}
\end{center}

\section{Acknowledgments}

We would like to thank the reviewers for their helpful comments.   Mike Sliczniak and Hasan Baran Kovuk
contributed to early parts of the SWILL implementation.   

\begin{thebibliography}{99}
\bibitem{beazley1} D.M.~Beazley and P.S.~Lomdahl, {\em Controlling the Data Glut in Large-Scale Molecular Dynamics Simulations}, Computers in Physics, Vol. 11, No. 3. (1997), p. 230-238. 

\bibitem{swig} D.M.~Beazley, {\em SWIG: An Easy to Use Tool for Integrating Scripting Languages with C and C++}, 4th Annual Tcl/Tk Workshop, Monterey, CA (1996).

\bibitem{apache} Apache Web Server, {\tt http://www.apache.org}.

\bibitem{zope} Zope, {\tt http://www.zope.org}.

\bibitem{soap} SOAP, {\tt http://www.w3.org/TR/SOAP/}.

\bibitem{gd} GD, {\tt http://www.boutell.com/gd/}.

\bibitem{xmlrpc} XML-RPC, {\tt http://www.xmlrpc.com}.

\bibitem{csp} C Server Pages,\\
 {\tt  http://tesitra.com/cserverpages}.

\bibitem{cgi} S.~Gundavaram, {\em CGI Programming}, O'Reilly \& Associates Inc., (1996).

\bibitem{yalnix} Yalnix placeholder (we have contacted Dave Johnson, the author, about a citation).

\bibitem{cactuspaper} G.~Allen et. al., {\em Supporting Efficient Execution in Heterogeneous Distributed Computing Environments with Cactus and Globus}, Supercomputing, Denver, CO, (2001).

\bibitem{cactus} Cactus Code, {\tt http://www.cactuscode.org}.

\bibitem{hsp} E.~Meijer and D.~van~Velzen,{\em Haskell Server Pages: Functional Programming and the Battle for the Middle Tier}, Proc. Haskell Workshop, (2000).

\bibitem{vetter} W.~Gu, J.~Vetter, K. Schwan. {\em Computational steering 
annotated bibliography},  Sigplan notices, 32 (6): 40-4, (1997). 

\bibitem {gsoap} R.A.~van~Engelen, K.A.~Gallivan, {\em The gSOAP Toolkit for Web Services and Peer-to-Peer Networks}, Proc. IEEE CC Grid Conf., (2002).

\bibitem{mpich} W.~Gropp, E.~Lusk, A~Skellum, {\em Using MPI: Portable Parallel Programming with the Message-Passing Interface}, MIT Press, (1999).

\bibitem{vetter2} J.~Vetter, K.~Schwan, {\em High Performance Computational 
Steering of Physical Simulations}, Proc. Int'l Parallel Processing Symp., 
Geneva, pp. 128-132, (1997). 

\end{thebibliography}

\end{document}
